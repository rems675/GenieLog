%%%%%%%%%%%%
%%        
%%   Préambule  
%%              

\documentclass[11pt]{report}
\usepackage[latin1]{inputenc}
\usepackage[british]{babel}
\usepackage[pdftex,pdfborder={0 0 0}]{hyperref}


\makeatletter
\renewcommand{\@chapapp}{}
\makeatother

\renewcommand{\thesection}{\arabic{section}}

%%%%%%%%%%%%
%%                      
%%   Document  
%%             
%%%%%%%%%%%%

 \begin{document}
% Informations du fichier pdf
\hypersetup{ % Modifiez la valeur des champs suivants
    pdfauthor   = {R\'{e}my SAKSIK},
    pdftitle    = {Other specifications},
    pdfsubject  = {Other specifications},
    pdfkeywords = {},
    pdfcreator  = {PDFLaTeX},
    pdfproducer = {PDFLaTeX}
    }
  %%%%%%%%%%%%
%  Page de titre 
%%%%%%%%%%%%
    \title{Other specifications}
    \author { 
    Etienne PAPEGNIES\\
    Remy SAKSIK\\
    Hamed SARGAZI\\
    Naima EL MISSOURI\\
    Avignon University\\
    France\\
}
   \maketitle
 % \cleardoublepage

 \setcounter{tocdepth}{1}
% \setcounter{secnumdepth}{1}



%%%%%%%%%%%%%%
% Table des matières 	
%%%%%%%%%%%%%%
    \tableofcontents
%%%%%%%%%%%%%%%%%%
    
   \chapter*{Input specifications}  
   \addcontentsline{toc}{chapter}{Input specifications} % pour ajouter l'introduction à la table des matières

  The user will have to specify the relatif path of the file.
  Or just the file's name if it's in the same repertory.
    \section{From command lines}
~~~~~~~The users will be respected in command lines for execute the program the next 
protype :

\begin{center}
 \begin{bf}"name of the program"\end{bf}
\end{center}

 \section{From extern file}

The content of file must be with this syntax :

\begin{center}
\begin{bf}"first number":"sign":"second number"\end{bf}
\end{center}
	
\begin{tiny}
    {\normalsize Voici Delimiters are} {\large EXTREMELY} {\normalsize important, it allows the program to identify.}
\end{tiny}

The content of file must have not more than one operation.

  \chapter*{Output specifications}  
   \addcontentsline{toc}{chapter}{Output specifications} % pour ajouter l'introduction à la table des matières

  \section{To a file}

    The user will have to specify the relatif path of the file.
Or just the file's name if it's in the same repertory.

Generate file with the resultat of the operation. 

\begin{bf}The content file will contain directly the result.\end{bf}

 \section{To a standard output}

The resultat will be showed on the screen.



\end{document}




