%%%%%%%%%%%%
%%        
%%   Préambule  
%%              

\documentclass[11pt]{report}
\usepackage[latin1]{inputenc}
\usepackage[british]{babel}
\usepackage[pdftex,pdfborder={0 0 0}]{hyperref}


\makeatletter
\renewcommand{\@chapapp}{}
\makeatother

\renewcommand{\thesection}{\arabic{section}}

%%%%%%%%%%%%
%%                      
%%   Document  
%%             
%%%%%%%%%%%%

 \begin{document}
% Informations du fichier pdf
\hypersetup{ % Modifiez la valeur des champs suivants
    pdfauthor   = {R\'{e}my SAKSIK},
    pdftitle    = {Technical specifications},
    pdfsubject  = {Technical specifications},
    pdfkeywords = {},
    pdfcreator  = {PDFLaTeX},
    pdfproducer = {PDFLaTeX}
    }
  %%%%%%%%%%%%
%  Page de titre 
%%%%%%%%%%%%
    \title{Technical specifications}
    \author { 
    Etienne PAPEGNIES\\
    Remy SAKSIK\\
    Hamed SARGAZI\\
    Naima EL MISSOURI\\
    Avignon University\\
    France\\
}
   \maketitle
 % \cleardoublepage

 \setcounter{tocdepth}{1}
% \setcounter{secnumdepth}{1}



%%%%%%%%%%%%%%
% Table des matières 	
%%%%%%%%%%%%%%
    \tableofcontents
%%%%%%%%%%%%%%%%%%
    
   \chapter*{Class specification}  
   \addcontentsline{toc}{chapter}{Class specification} % pour ajouter l'introduction à la table des matières
   
	\begin{bf}\underline{Here is the description of class method LongInt.}\end{bf}
    \section{LongInt()}
	\subsection{Summary}
~~~~~~~~~~~The role of the default constructor is to retrieve data users.
	\subsection{Function Parameters}
~~~~~~~~~~~No parameters.
	\subsection{Return Value}
~~~~~~~~~~~No return value.
	    
    \section{LongInt(int)}
   
	\subsection{Summary}
~~~~~~~~~~~The role of this constructor is to simply create an empty object. (Useful for multiplication)
	\subsection{Function Parameters}
~~~~~~~~~~~No parameters.
	\subsection{Return Value}
~~~~~~~~~~~No return value.
    
    \section{void display()}
  
	\subsection{Summary}
~~~~~~~~~~~The role of this function and allow a display of the result on stdout\\
~~~~~~~~~~~or into a file.
	\subsection{Function Parameters}
~~~~~~~~~~~No parameters.
	\subsection{Return Value}
~~~~~~~~~~~No return value.
	  
    \section{void expand(int)}
  
	\subsection{Summary}
~~~~~~~~~~~The role of this function is to reserve memory space aditional.
	\subsection{Function Parameters}
~~~~~~~~~~~\begin{bf}int\end{bf}: Represents the space needs.
	\subsection{Return Value}
~~~~~~~~~~~No return value.
	  
    \section{inc()}
    
	\subsection{Summary}
~~~~~~~~~~~The role of this function and allow incrementing a LongInt.
	\subsection{Function Parameters}
~~~~~~~~~~~No parameters.
	\subsection{Return Value}
~~~~~~~~~~~No return value.
	
     \section{dec()}
	\subsection{Summary}
~~~~~~~~~~~The role of this function and allow decrementing a LongInt.
	\subsection{Function Parameters}
~~~~~~~~~~~No parameters. 
	\subsection{Return Value}
~~~~~~~~~~~No return value.

      \section{void add(LongInt \&)}
	\subsection{Summary}
~~~~~~~~~~~The role of this function is to manage the number of sign to call the function that performs the addition "coreAdd" (specified farther down).
	\subsection{Function Parameters}
~~~~~~~~~~~\begin{bf}LongInt\&\end{bf}: Is the object pass by reference which contains the number by whichare to be added to the figure that was invoked on the object.

One reference, unwanted because the parameter must not be modified and we must be able to access directly.
	\subsection{Return Value}
~~~~~~~~~~~No return value.

\section{void sub(LongInt \&)}
	\subsection{Summary}
~~~~~~~~~~~The role of this function is to book a table in memory that contain the result of the subtraction
and also to manage the sign.
To call the function that performs the subtraction "coreSub" (specified farther down)
	\subsection{Function Parameters}
~~~~~~~~~~~\begin{bf}LongInt\&\end{bf}: Is the object pass by reference which contains the digit number which
we want to additioner figure that was invoked on the local object.

One reference, unwanted because the parameter must not be modified and
we must be able to access directly.

	\subsection{Return Value}
~~~~~~~~~~~No return value.


\section{void mul(LongInt \&)}
	\subsection{Summary}
~~~~~~~~~~~The role of this function is to manage two numbers and
the sign and the size of the array that will contain the result.
calling CoreMul to perform multiplication, then sum the result for coreAdd.
	\subsection{Function Parameters}
~~~~~~~~~~~\begin{bf}LongInt\&\end{bf}: Is the object pass by reference which contains the digit number by which
we want to multiply that figure was invoked on the object.

One reference, unwanted because the parameter must not be modified and
we must be able to access directly.

\subsection{Return Value}
~~~~~~~~~~~No return value.

\section{void div(LongInt \&)}
	\subsection{Summary}
~~~~~~~~~~~The role of this function is to manage two numbers and
the sign and the size of the array that will contain the result.
calling CoreMul to perform multiplication, then sum the result for coreAdd.
	\subsection{Function Parameters}
~~~~~~~~~~~\begin{bf}LongInt\&\end{bf}: Is the object pass by reference which contains the digit number by which
we want to multiply that figure was invoked on the object.

One reference, unwanted because the parameter must not be modified and
we must be able to access directly.

\subsection{Return Value}
~~~~~~~~~~~No return value.

    \chapter*{Arithmetic operations}  
    \addcontentsline{toc}{chapter}{Arithmetic operations} % pour ajouter l'introduction à la table des matières

      \begin{bf}
~~~We introduice with a little vocabulary
a.add(b) correspond to: \begin{center}  local = a~~~and~~~distant = b. \end{center}
\begin{center}Modify is local but not remote.\end{center}
    \end{bf}


\section{void coreAdd(LongInt \&)}
	\subsection{Summary}
~~~~~~~~~~~Perform the addition of two big positive integers.
divide the number that was invoked on the object.
One reference, unwanted because the parameter must not be modified and
we must be able to access directly.
	\subsection{Function Parameters}
~~~~~~~~~~~\begin{bf}LongInt\&\end{bf}: Perform the addition the number passed in parameter to the local number.

\subsection{Return Value}
~~~~~~~~~~~No return value.

\section{void coreSub(LongInt \&)}
	\subsection{Summary}
~~~~~~~~~~~The role of this function is to perform the subtraction of two digit.
~~~~~~~~~~~Subtract that was invoked on the object.
One reference, unwanted because the parameter must not be modified and
we must be able to access directly.
	\subsection{Function Parameters}
~~~~~~~~~~~\begin{bf}LongInt\&\end{bf}: Perform the substraction the number passed in parameter to the local number.

	\subsection{Return Value}
~~~~~~~~~~~No return value.

\section{void coreMul(LongInt \&, char, int)}
	\subsection{Summary}
~~~~~~~~~~~The role of this function and multiply the figure in question by a number of multiplier
and store the result in a temporary array.
	\subsection{Function Parameters}
	\begin{tabbing} 
\begin{bf}LongInt\&\end{bf}: Is the target object of a temporary operation.\\
\begin{bf}char\end{bf}: Corresponds to a digit of the multiplier.\\
\begin{bf}int\end{bf}: Represents the number of zero are added to the right\\
correspond to the correct power of 10.
	\end{tabbing} 

\section{void coreDiv(LongInt \& N, LongInt \& Quotient)}
	\subsection{Summary}
~~~~~~~~~~~ Divide local by distant object N, N being at most ten times local. Local is set with the remainder,
 the actual result is set in Quotient
	\subsection{Function Parameters}
	\begin{tabbing} 
\begin{bf}LongInt\& N \end{bf}: Is the object that represent the remote object with the remainder\\
\begin{bf}LongInt\&  Quotient \end{bf}: Is the target object of the result.\\
	\end{tabbing} 

\subsection{Return Value}
~~~~~~~~~~~No return value.

\end{document}




